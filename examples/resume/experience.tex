%-------------------------------------------------------------------------------
%	SECTION TITLE
%-------------------------------------------------------------------------------
\cvsection{工作经验}


%-------------------------------------------------------------------------------
%	CONTENT
%-------------------------------------------------------------------------------
\begin{cventries}

%---------------------------------------------------------
  \cventry
    {资深开发工程师} % Job title
    {蘑菇街} % Organization
    {} % Location
    {2016.03 - 至今} % Date(s)
    {
      \begin{cvitems} % Description(s) of tasks/responsibilities
        \item {2017.01 - 2017.03 上线NoSQL服务。提供Mongodb、Cassandra线上服务,为业务提供更多的可能性;使用Ansible Playbook完成半自化集群部署;使用YCSB针对用户场景压测产出数据;开发监控程序汇报数据到内部监控产品。}
        \item {2017.02 - 2017.03 主导Java应用容器方案。主导类Pandora应用容器方案设计和Demo实现展示,为解决应用Jar包冲突和中间件自动升级打下了基础。}
        \item {2016.10 - 2016.12 ZooKeeper管理系统开发。结合现有开源ZooKeeper管理系统TaoKeeper和Exibitor的优势,使用Spring MVC + Angularjs开发,提供节点状态展示、数据查看、 配置对比等功能。结合ZooKeeper源码解决维护过程中的问题。}
        \item {2016.09 - 2016.10 引入Redis Sentinel实现用户系统缓存HA。基于业务系统使用现状调研Redis HA方案,通过评审,指导业务方改动落地;上线之后进行容灾演练并优化配置;将单点故障对业务的影响时间从数十分钟降低到10s}
        \item {2016.07 - 2016.09 分布式产品自动化容灾测试。引入Jepsen分布式容灾测试框架,暴露了一些之前测试方法很难暴露的问题,降低了容灾测试的成本;有效测试出了内部分布式Redis产品扩容过程中会产生一致性问题的风险;利用Docker Machine在本机搭建分布式环境,提高测试效率。}
        \item {2016.04 - 2016.06 基于HDFS Erasure Code实现交易快照冷数据存储方案。交易快照冷热数据分离的一部分,在使用Erasure Code的HDFS集群之上搭建HBase,提供冷数据服务,有效地将存储成本降低了25\%;使用工具将Mysql中的数据导出到Hbase集群中;解决应用批量查询Rt过长的问题。}
      \end{cvitems}
    }

%---------------------------------------------------------
  \cventry
    {资深开发工程师} % Job title
    {阿里巴巴} % Organization
    {} % Location
    {2012.04 - 2015.12} % Date(s)
    {
      \begin{cvitems} % Description(s) of tasks/responsibilities
        \item {Lead engineer on agent-less backtracking system that can discover client device's fingerprint(including public and private IP) independently of the Proxy, VPN and NAT.}
        \item {Implemented a distributed web stress test tool with high anonymity.}
        \item {Implemented a military cooperation system which is web based real time messenger in Scala on Lift.}
      \end{cvitems}
    }

%---------------------------------------------------------
  \cventry
    {实习生} % Job title
    {微软亚洲搜索技术中心} % Organization
    {北京} % Location
    {2011.07 - 2011.09} % Date(s)
    {
      \begin{cvitems} % Description(s) of tasks/responsibilities
        \item {开发工具综合索引数据和用户点击数据进行处理,在多级粒度上对站点行为进行评估建模;利用建模结果中索引覆盖率等相关指标对索引进行诊断,并从中得到对搜索结果排序的信号;用近似方法判断页面是否是新闻页面}
      \end{cvitems}
    }

%---------------------------------------------------------
\end{cventries}
